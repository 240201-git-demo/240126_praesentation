\documentclass[table]{beamer}
\usepackage[german]{babel}

\usepackage{tikz}
\usetikzlibrary{positioning,
                calc,
                decorations.pathreplacing,
                calligraphy}
\usepackage{tikzscale}
\usepackage{libertine}
\usepackage{pgf-pie}
\usepackage{hyperref}
\usepackage{booktabs}
\usepackage{diagbox}
\usepackage{xcolor}
\usepackage{listings}
\lstset{
    backgroundcolor=\color{black},
    basicstyle=\footnotesize,
    basicstyle=\color{white},
    breaklines=true,
    xleftmargin=-2cm,
    framesep=5pt,
    keywordstyle=\color{blue}
}

\title{Versionskontrolle von Texten}
\subtitle{Git und GitHub}
\date{1. Februar 2024}

\begin{document}
    \frame{\titlepage}

    \begin{frame}
        \frametitle{Wichtige Quellen}
        \begin{itemize}
            \item git-scm.com
            \item docs.github.com/decorations
            \item code.visualstudio.com/docs/sourcecontrol/overview
        \end{itemize}
    \end{frame}

    \begin{frame}
        \frametitle{Ausgangslage}
        Wer kennt das nicht?

        \vspace*{5mm}

       \only<2>{\includegraphics[width=\textwidth]{images/word_markup.png}} 
       \only<3>{\includegraphics[width=\textwidth]{images/file_manager.png}}
    \end{frame}

    \begin{frame}
        \frametitle{Ausgangslage}
        Das hoffentlich nicht, aber wer weiss?

        \vspace*{5mm}

        \visible<2>{\includegraphics[width=\textwidth]{images/burning_laptop.png}} 
       
    \end{frame}

    \begin{frame}
        \frametitle{Warum Versionierung}

        \begin{itemize}
            \item Back-up
            \item just one source of truth
            \item transparente Textgenese
        \end{itemize}
    \end{frame}

    \begin{frame}
        \frametitle{Back-up}

        \begin{itemize}
            \item lokales Back-up
                
            nicht so sicher -- insbesondere im Falle eines Velusts des
            Computers
            
           \item Back-up in der Cloud
           \begin{itemize}
            \item GitHub
            \item GitLab
            \item GitBucket
            \item Azure Repos
            \item \dots
           \end{itemize}
        \end{itemize}   
        
    
    \end{frame}

    \begin{frame}
        \frametitle{Just \textbf{One} Source of Truth}

        \visible<2>{\includegraphics[width=\textwidth]{images/git_graph_details.png}}
  
        
    
    \end{frame}

    \begin{frame}
        \frametitle{Transparente Textgenese}

        \visible<2>{\includegraphics[width=\textwidth]{images/git_graph_aenderung.png}}
    
        
    
    \end{frame}

    \begin{frame}
        \frametitle{Erforderliche Vorbereitungen}

        \begin{itemize}
            \item \textbf{Git (Software)}
            \item GitHub (Konto)
            \item Visual Studio Code (Editor)
            \item pandoc
        \end{itemize}
    
    \end{frame}

    \begin{frame}
        \frametitle{Git}

        \begin{itemize}
            \item Git ist \textbf{die} Versionierungssoftware
            \item Download von git-scm.com 
        \end{itemize}
    
    \end{frame}

    \begin{frame}
        \frametitle{GitHub Konto}

        \begin{itemize}
            \item github.com 
            \item registrieren mit KBW-Identität
        \end{itemize}
    \end{frame}

    \begin{frame}
        \frametitle{Visual Studio Code}

        \begin{itemize}
            \item Editor (nicht erforderlich, aber komfortabel)
            \item Download von code.visualstudio.com/download 
            \item Erweiterung Git Graph (ev. Rewrap, Spell Right, German
            Language Pack for Visual Studio Code)
        \end{itemize}
    
    \end{frame}

    \begin{frame}
        \frametitle{pandoc}

        \begin{itemize}
            \item Format Konverter (nicht erforderlich, aber hilfreich)
            \item Download von pandoc.org/installing.html
        \end{itemize}   
            
    \end{frame}

    \begin{frame}[fragile]
        \frametitle{Mein erstes Projekt}

        Vorbereitung:

        \begin{lstlisting}[language=Bash]
            git config --global user.name = "John Doe"

            git config --global user.email = john.doe@kbw.ch
        \end{lstlisting}

        
    
        
    
    \end{frame}

    
        
    
    



\end{document}
